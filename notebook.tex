
% Default to the notebook output style

    


% Inherit from the specified cell style.




    
\documentclass[11pt]{article}

    
    
    \usepackage[T1]{fontenc}
    % Nicer default font (+ math font) than Computer Modern for most use cases
    \usepackage{mathpazo}

    % Basic figure setup, for now with no caption control since it's done
    % automatically by Pandoc (which extracts ![](path) syntax from Markdown).
    \usepackage{graphicx}
    % We will generate all images so they have a width \maxwidth. This means
    % that they will get their normal width if they fit onto the page, but
    % are scaled down if they would overflow the margins.
    \makeatletter
    \def\maxwidth{\ifdim\Gin@nat@width>\linewidth\linewidth
    \else\Gin@nat@width\fi}
    \makeatother
    \let\Oldincludegraphics\includegraphics
    % Set max figure width to be 80% of text width, for now hardcoded.
    \renewcommand{\includegraphics}[1]{\Oldincludegraphics[width=.8\maxwidth]{#1}}
    % Ensure that by default, figures have no caption (until we provide a
    % proper Figure object with a Caption API and a way to capture that
    % in the conversion process - todo).
    \usepackage{caption}
    \DeclareCaptionLabelFormat{nolabel}{}
    \captionsetup{labelformat=nolabel}

    \usepackage{adjustbox} % Used to constrain images to a maximum size 
    \usepackage{xcolor} % Allow colors to be defined
    \usepackage{enumerate} % Needed for markdown enumerations to work
    \usepackage{geometry} % Used to adjust the document margins
    \usepackage{amsmath} % Equations
    \usepackage{amssymb} % Equations
    \usepackage{textcomp} % defines textquotesingle
    % Hack from http://tex.stackexchange.com/a/47451/13684:
    \AtBeginDocument{%
        \def\PYZsq{\textquotesingle}% Upright quotes in Pygmentized code
    }
    \usepackage{upquote} % Upright quotes for verbatim code
    \usepackage{eurosym} % defines \euro
    \usepackage[mathletters]{ucs} % Extended unicode (utf-8) support
    \usepackage[utf8x]{inputenc} % Allow utf-8 characters in the tex document
    \usepackage{fancyvrb} % verbatim replacement that allows latex
    \usepackage{grffile} % extends the file name processing of package graphics 
                         % to support a larger range 
    % The hyperref package gives us a pdf with properly built
    % internal navigation ('pdf bookmarks' for the table of contents,
    % internal cross-reference links, web links for URLs, etc.)
    \usepackage{hyperref}
    \usepackage{longtable} % longtable support required by pandoc >1.10
    \usepackage{booktabs}  % table support for pandoc > 1.12.2
    \usepackage[inline]{enumitem} % IRkernel/repr support (it uses the enumerate* environment)
    \usepackage[normalem]{ulem} % ulem is needed to support strikethroughs (\sout)
                                % normalem makes italics be italics, not underlines
    

    
    
    % Colors for the hyperref package
    \definecolor{urlcolor}{rgb}{0,.145,.698}
    \definecolor{linkcolor}{rgb}{.71,0.21,0.01}
    \definecolor{citecolor}{rgb}{.12,.54,.11}

    % ANSI colors
    \definecolor{ansi-black}{HTML}{3E424D}
    \definecolor{ansi-black-intense}{HTML}{282C36}
    \definecolor{ansi-red}{HTML}{E75C58}
    \definecolor{ansi-red-intense}{HTML}{B22B31}
    \definecolor{ansi-green}{HTML}{00A250}
    \definecolor{ansi-green-intense}{HTML}{007427}
    \definecolor{ansi-yellow}{HTML}{DDB62B}
    \definecolor{ansi-yellow-intense}{HTML}{B27D12}
    \definecolor{ansi-blue}{HTML}{208FFB}
    \definecolor{ansi-blue-intense}{HTML}{0065CA}
    \definecolor{ansi-magenta}{HTML}{D160C4}
    \definecolor{ansi-magenta-intense}{HTML}{A03196}
    \definecolor{ansi-cyan}{HTML}{60C6C8}
    \definecolor{ansi-cyan-intense}{HTML}{258F8F}
    \definecolor{ansi-white}{HTML}{C5C1B4}
    \definecolor{ansi-white-intense}{HTML}{A1A6B2}

    % commands and environments needed by pandoc snippets
    % extracted from the output of `pandoc -s`
    \providecommand{\tightlist}{%
      \setlength{\itemsep}{0pt}\setlength{\parskip}{0pt}}
    \DefineVerbatimEnvironment{Highlighting}{Verbatim}{commandchars=\\\{\}}
    % Add ',fontsize=\small' for more characters per line
    \newenvironment{Shaded}{}{}
    \newcommand{\KeywordTok}[1]{\textcolor[rgb]{0.00,0.44,0.13}{\textbf{{#1}}}}
    \newcommand{\DataTypeTok}[1]{\textcolor[rgb]{0.56,0.13,0.00}{{#1}}}
    \newcommand{\DecValTok}[1]{\textcolor[rgb]{0.25,0.63,0.44}{{#1}}}
    \newcommand{\BaseNTok}[1]{\textcolor[rgb]{0.25,0.63,0.44}{{#1}}}
    \newcommand{\FloatTok}[1]{\textcolor[rgb]{0.25,0.63,0.44}{{#1}}}
    \newcommand{\CharTok}[1]{\textcolor[rgb]{0.25,0.44,0.63}{{#1}}}
    \newcommand{\StringTok}[1]{\textcolor[rgb]{0.25,0.44,0.63}{{#1}}}
    \newcommand{\CommentTok}[1]{\textcolor[rgb]{0.38,0.63,0.69}{\textit{{#1}}}}
    \newcommand{\OtherTok}[1]{\textcolor[rgb]{0.00,0.44,0.13}{{#1}}}
    \newcommand{\AlertTok}[1]{\textcolor[rgb]{1.00,0.00,0.00}{\textbf{{#1}}}}
    \newcommand{\FunctionTok}[1]{\textcolor[rgb]{0.02,0.16,0.49}{{#1}}}
    \newcommand{\RegionMarkerTok}[1]{{#1}}
    \newcommand{\ErrorTok}[1]{\textcolor[rgb]{1.00,0.00,0.00}{\textbf{{#1}}}}
    \newcommand{\NormalTok}[1]{{#1}}
    
    % Additional commands for more recent versions of Pandoc
    \newcommand{\ConstantTok}[1]{\textcolor[rgb]{0.53,0.00,0.00}{{#1}}}
    \newcommand{\SpecialCharTok}[1]{\textcolor[rgb]{0.25,0.44,0.63}{{#1}}}
    \newcommand{\VerbatimStringTok}[1]{\textcolor[rgb]{0.25,0.44,0.63}{{#1}}}
    \newcommand{\SpecialStringTok}[1]{\textcolor[rgb]{0.73,0.40,0.53}{{#1}}}
    \newcommand{\ImportTok}[1]{{#1}}
    \newcommand{\DocumentationTok}[1]{\textcolor[rgb]{0.73,0.13,0.13}{\textit{{#1}}}}
    \newcommand{\AnnotationTok}[1]{\textcolor[rgb]{0.38,0.63,0.69}{\textbf{\textit{{#1}}}}}
    \newcommand{\CommentVarTok}[1]{\textcolor[rgb]{0.38,0.63,0.69}{\textbf{\textit{{#1}}}}}
    \newcommand{\VariableTok}[1]{\textcolor[rgb]{0.10,0.09,0.49}{{#1}}}
    \newcommand{\ControlFlowTok}[1]{\textcolor[rgb]{0.00,0.44,0.13}{\textbf{{#1}}}}
    \newcommand{\OperatorTok}[1]{\textcolor[rgb]{0.40,0.40,0.40}{{#1}}}
    \newcommand{\BuiltInTok}[1]{{#1}}
    \newcommand{\ExtensionTok}[1]{{#1}}
    \newcommand{\PreprocessorTok}[1]{\textcolor[rgb]{0.74,0.48,0.00}{{#1}}}
    \newcommand{\AttributeTok}[1]{\textcolor[rgb]{0.49,0.56,0.16}{{#1}}}
    \newcommand{\InformationTok}[1]{\textcolor[rgb]{0.38,0.63,0.69}{\textbf{\textit{{#1}}}}}
    \newcommand{\WarningTok}[1]{\textcolor[rgb]{0.38,0.63,0.69}{\textbf{\textit{{#1}}}}}
    
    
    % Define a nice break command that doesn't care if a line doesn't already
    % exist.
    \def\br{\hspace*{\fill} \\* }
    % Math Jax compatability definitions
    \def\gt{>}
    \def\lt{<}
    % Document parameters
    \title{investigate-a-dataset-template}
    
    
    

    % Pygments definitions
    
\makeatletter
\def\PY@reset{\let\PY@it=\relax \let\PY@bf=\relax%
    \let\PY@ul=\relax \let\PY@tc=\relax%
    \let\PY@bc=\relax \let\PY@ff=\relax}
\def\PY@tok#1{\csname PY@tok@#1\endcsname}
\def\PY@toks#1+{\ifx\relax#1\empty\else%
    \PY@tok{#1}\expandafter\PY@toks\fi}
\def\PY@do#1{\PY@bc{\PY@tc{\PY@ul{%
    \PY@it{\PY@bf{\PY@ff{#1}}}}}}}
\def\PY#1#2{\PY@reset\PY@toks#1+\relax+\PY@do{#2}}

\expandafter\def\csname PY@tok@w\endcsname{\def\PY@tc##1{\textcolor[rgb]{0.73,0.73,0.73}{##1}}}
\expandafter\def\csname PY@tok@c\endcsname{\let\PY@it=\textit\def\PY@tc##1{\textcolor[rgb]{0.25,0.50,0.50}{##1}}}
\expandafter\def\csname PY@tok@cp\endcsname{\def\PY@tc##1{\textcolor[rgb]{0.74,0.48,0.00}{##1}}}
\expandafter\def\csname PY@tok@k\endcsname{\let\PY@bf=\textbf\def\PY@tc##1{\textcolor[rgb]{0.00,0.50,0.00}{##1}}}
\expandafter\def\csname PY@tok@kp\endcsname{\def\PY@tc##1{\textcolor[rgb]{0.00,0.50,0.00}{##1}}}
\expandafter\def\csname PY@tok@kt\endcsname{\def\PY@tc##1{\textcolor[rgb]{0.69,0.00,0.25}{##1}}}
\expandafter\def\csname PY@tok@o\endcsname{\def\PY@tc##1{\textcolor[rgb]{0.40,0.40,0.40}{##1}}}
\expandafter\def\csname PY@tok@ow\endcsname{\let\PY@bf=\textbf\def\PY@tc##1{\textcolor[rgb]{0.67,0.13,1.00}{##1}}}
\expandafter\def\csname PY@tok@nb\endcsname{\def\PY@tc##1{\textcolor[rgb]{0.00,0.50,0.00}{##1}}}
\expandafter\def\csname PY@tok@nf\endcsname{\def\PY@tc##1{\textcolor[rgb]{0.00,0.00,1.00}{##1}}}
\expandafter\def\csname PY@tok@nc\endcsname{\let\PY@bf=\textbf\def\PY@tc##1{\textcolor[rgb]{0.00,0.00,1.00}{##1}}}
\expandafter\def\csname PY@tok@nn\endcsname{\let\PY@bf=\textbf\def\PY@tc##1{\textcolor[rgb]{0.00,0.00,1.00}{##1}}}
\expandafter\def\csname PY@tok@ne\endcsname{\let\PY@bf=\textbf\def\PY@tc##1{\textcolor[rgb]{0.82,0.25,0.23}{##1}}}
\expandafter\def\csname PY@tok@nv\endcsname{\def\PY@tc##1{\textcolor[rgb]{0.10,0.09,0.49}{##1}}}
\expandafter\def\csname PY@tok@no\endcsname{\def\PY@tc##1{\textcolor[rgb]{0.53,0.00,0.00}{##1}}}
\expandafter\def\csname PY@tok@nl\endcsname{\def\PY@tc##1{\textcolor[rgb]{0.63,0.63,0.00}{##1}}}
\expandafter\def\csname PY@tok@ni\endcsname{\let\PY@bf=\textbf\def\PY@tc##1{\textcolor[rgb]{0.60,0.60,0.60}{##1}}}
\expandafter\def\csname PY@tok@na\endcsname{\def\PY@tc##1{\textcolor[rgb]{0.49,0.56,0.16}{##1}}}
\expandafter\def\csname PY@tok@nt\endcsname{\let\PY@bf=\textbf\def\PY@tc##1{\textcolor[rgb]{0.00,0.50,0.00}{##1}}}
\expandafter\def\csname PY@tok@nd\endcsname{\def\PY@tc##1{\textcolor[rgb]{0.67,0.13,1.00}{##1}}}
\expandafter\def\csname PY@tok@s\endcsname{\def\PY@tc##1{\textcolor[rgb]{0.73,0.13,0.13}{##1}}}
\expandafter\def\csname PY@tok@sd\endcsname{\let\PY@it=\textit\def\PY@tc##1{\textcolor[rgb]{0.73,0.13,0.13}{##1}}}
\expandafter\def\csname PY@tok@si\endcsname{\let\PY@bf=\textbf\def\PY@tc##1{\textcolor[rgb]{0.73,0.40,0.53}{##1}}}
\expandafter\def\csname PY@tok@se\endcsname{\let\PY@bf=\textbf\def\PY@tc##1{\textcolor[rgb]{0.73,0.40,0.13}{##1}}}
\expandafter\def\csname PY@tok@sr\endcsname{\def\PY@tc##1{\textcolor[rgb]{0.73,0.40,0.53}{##1}}}
\expandafter\def\csname PY@tok@ss\endcsname{\def\PY@tc##1{\textcolor[rgb]{0.10,0.09,0.49}{##1}}}
\expandafter\def\csname PY@tok@sx\endcsname{\def\PY@tc##1{\textcolor[rgb]{0.00,0.50,0.00}{##1}}}
\expandafter\def\csname PY@tok@m\endcsname{\def\PY@tc##1{\textcolor[rgb]{0.40,0.40,0.40}{##1}}}
\expandafter\def\csname PY@tok@gh\endcsname{\let\PY@bf=\textbf\def\PY@tc##1{\textcolor[rgb]{0.00,0.00,0.50}{##1}}}
\expandafter\def\csname PY@tok@gu\endcsname{\let\PY@bf=\textbf\def\PY@tc##1{\textcolor[rgb]{0.50,0.00,0.50}{##1}}}
\expandafter\def\csname PY@tok@gd\endcsname{\def\PY@tc##1{\textcolor[rgb]{0.63,0.00,0.00}{##1}}}
\expandafter\def\csname PY@tok@gi\endcsname{\def\PY@tc##1{\textcolor[rgb]{0.00,0.63,0.00}{##1}}}
\expandafter\def\csname PY@tok@gr\endcsname{\def\PY@tc##1{\textcolor[rgb]{1.00,0.00,0.00}{##1}}}
\expandafter\def\csname PY@tok@ge\endcsname{\let\PY@it=\textit}
\expandafter\def\csname PY@tok@gs\endcsname{\let\PY@bf=\textbf}
\expandafter\def\csname PY@tok@gp\endcsname{\let\PY@bf=\textbf\def\PY@tc##1{\textcolor[rgb]{0.00,0.00,0.50}{##1}}}
\expandafter\def\csname PY@tok@go\endcsname{\def\PY@tc##1{\textcolor[rgb]{0.53,0.53,0.53}{##1}}}
\expandafter\def\csname PY@tok@gt\endcsname{\def\PY@tc##1{\textcolor[rgb]{0.00,0.27,0.87}{##1}}}
\expandafter\def\csname PY@tok@err\endcsname{\def\PY@bc##1{\setlength{\fboxsep}{0pt}\fcolorbox[rgb]{1.00,0.00,0.00}{1,1,1}{\strut ##1}}}
\expandafter\def\csname PY@tok@kc\endcsname{\let\PY@bf=\textbf\def\PY@tc##1{\textcolor[rgb]{0.00,0.50,0.00}{##1}}}
\expandafter\def\csname PY@tok@kd\endcsname{\let\PY@bf=\textbf\def\PY@tc##1{\textcolor[rgb]{0.00,0.50,0.00}{##1}}}
\expandafter\def\csname PY@tok@kn\endcsname{\let\PY@bf=\textbf\def\PY@tc##1{\textcolor[rgb]{0.00,0.50,0.00}{##1}}}
\expandafter\def\csname PY@tok@kr\endcsname{\let\PY@bf=\textbf\def\PY@tc##1{\textcolor[rgb]{0.00,0.50,0.00}{##1}}}
\expandafter\def\csname PY@tok@bp\endcsname{\def\PY@tc##1{\textcolor[rgb]{0.00,0.50,0.00}{##1}}}
\expandafter\def\csname PY@tok@fm\endcsname{\def\PY@tc##1{\textcolor[rgb]{0.00,0.00,1.00}{##1}}}
\expandafter\def\csname PY@tok@vc\endcsname{\def\PY@tc##1{\textcolor[rgb]{0.10,0.09,0.49}{##1}}}
\expandafter\def\csname PY@tok@vg\endcsname{\def\PY@tc##1{\textcolor[rgb]{0.10,0.09,0.49}{##1}}}
\expandafter\def\csname PY@tok@vi\endcsname{\def\PY@tc##1{\textcolor[rgb]{0.10,0.09,0.49}{##1}}}
\expandafter\def\csname PY@tok@vm\endcsname{\def\PY@tc##1{\textcolor[rgb]{0.10,0.09,0.49}{##1}}}
\expandafter\def\csname PY@tok@sa\endcsname{\def\PY@tc##1{\textcolor[rgb]{0.73,0.13,0.13}{##1}}}
\expandafter\def\csname PY@tok@sb\endcsname{\def\PY@tc##1{\textcolor[rgb]{0.73,0.13,0.13}{##1}}}
\expandafter\def\csname PY@tok@sc\endcsname{\def\PY@tc##1{\textcolor[rgb]{0.73,0.13,0.13}{##1}}}
\expandafter\def\csname PY@tok@dl\endcsname{\def\PY@tc##1{\textcolor[rgb]{0.73,0.13,0.13}{##1}}}
\expandafter\def\csname PY@tok@s2\endcsname{\def\PY@tc##1{\textcolor[rgb]{0.73,0.13,0.13}{##1}}}
\expandafter\def\csname PY@tok@sh\endcsname{\def\PY@tc##1{\textcolor[rgb]{0.73,0.13,0.13}{##1}}}
\expandafter\def\csname PY@tok@s1\endcsname{\def\PY@tc##1{\textcolor[rgb]{0.73,0.13,0.13}{##1}}}
\expandafter\def\csname PY@tok@mb\endcsname{\def\PY@tc##1{\textcolor[rgb]{0.40,0.40,0.40}{##1}}}
\expandafter\def\csname PY@tok@mf\endcsname{\def\PY@tc##1{\textcolor[rgb]{0.40,0.40,0.40}{##1}}}
\expandafter\def\csname PY@tok@mh\endcsname{\def\PY@tc##1{\textcolor[rgb]{0.40,0.40,0.40}{##1}}}
\expandafter\def\csname PY@tok@mi\endcsname{\def\PY@tc##1{\textcolor[rgb]{0.40,0.40,0.40}{##1}}}
\expandafter\def\csname PY@tok@il\endcsname{\def\PY@tc##1{\textcolor[rgb]{0.40,0.40,0.40}{##1}}}
\expandafter\def\csname PY@tok@mo\endcsname{\def\PY@tc##1{\textcolor[rgb]{0.40,0.40,0.40}{##1}}}
\expandafter\def\csname PY@tok@ch\endcsname{\let\PY@it=\textit\def\PY@tc##1{\textcolor[rgb]{0.25,0.50,0.50}{##1}}}
\expandafter\def\csname PY@tok@cm\endcsname{\let\PY@it=\textit\def\PY@tc##1{\textcolor[rgb]{0.25,0.50,0.50}{##1}}}
\expandafter\def\csname PY@tok@cpf\endcsname{\let\PY@it=\textit\def\PY@tc##1{\textcolor[rgb]{0.25,0.50,0.50}{##1}}}
\expandafter\def\csname PY@tok@c1\endcsname{\let\PY@it=\textit\def\PY@tc##1{\textcolor[rgb]{0.25,0.50,0.50}{##1}}}
\expandafter\def\csname PY@tok@cs\endcsname{\let\PY@it=\textit\def\PY@tc##1{\textcolor[rgb]{0.25,0.50,0.50}{##1}}}

\def\PYZbs{\char`\\}
\def\PYZus{\char`\_}
\def\PYZob{\char`\{}
\def\PYZcb{\char`\}}
\def\PYZca{\char`\^}
\def\PYZam{\char`\&}
\def\PYZlt{\char`\<}
\def\PYZgt{\char`\>}
\def\PYZsh{\char`\#}
\def\PYZpc{\char`\%}
\def\PYZdl{\char`\$}
\def\PYZhy{\char`\-}
\def\PYZsq{\char`\'}
\def\PYZdq{\char`\"}
\def\PYZti{\char`\~}
% for compatibility with earlier versions
\def\PYZat{@}
\def\PYZlb{[}
\def\PYZrb{]}
\makeatother


    % Exact colors from NB
    \definecolor{incolor}{rgb}{0.0, 0.0, 0.5}
    \definecolor{outcolor}{rgb}{0.545, 0.0, 0.0}



    
    % Prevent overflowing lines due to hard-to-break entities
    \sloppy 
    % Setup hyperref package
    \hypersetup{
      breaklinks=true,  % so long urls are correctly broken across lines
      colorlinks=true,
      urlcolor=urlcolor,
      linkcolor=linkcolor,
      citecolor=citecolor,
      }
    % Slightly bigger margins than the latex defaults
    
    \geometry{verbose,tmargin=1in,bmargin=1in,lmargin=1in,rmargin=1in}
    
    

    \begin{document}
    
    
    \maketitle
    
    

    
    \subsubsection{This is my rough draft of the project. Please let me know
if I am on the right track. This project is challenging for
me}\label{this-is-my-rough-draft-of-the-project.-please-let-me-know-if-i-am-on-the-right-track.-this-project-is-challenging-for-me}

    \begin{quote}
\textbf{Tip}: Welcome to the Investigate a Dataset project! You will
find tips in quoted sections like this to help organize your approach to
your investigation. Before submitting your project, it will be a good
idea to go back through your report and remove these sections to make
the presentation of your work as tidy as possible. First things first,
you might want to double-click this Markdown cell and change the title
so that it reflects your dataset and investigation.
\end{quote}

\section{Project: Investigate a Dataset (Replace this with something
more
specific!)}\label{project-investigate-a-dataset-replace-this-with-something-more-specific}

\subsection{Table of Contents}\label{table-of-contents}

Introduction

Data Wrangling

Exploratory Data Analysis

Conclusions

     \#\# Introduction

\begin{quote}
\textbf{Tip}: In this section of the report, provide a brief
introduction to the dataset you've selected for analysis. At the end of
this section, describe the questions that you plan on exploring over the
course of the report. Try to build your report around the analysis of at
least one dependent variable and three independent variables.

If you haven't yet selected and downloaded your data, make sure you do
that first before coming back here. If you're not sure what questions to
ask right now, then make sure you familiarize yourself with the
variables and the dataset context for ideas of what to explore.
\end{quote}

    \begin{Verbatim}[commandchars=\\\{\}]
{\color{incolor}In [{\color{incolor}14}]:} \PY{c+c1}{\PYZsh{} Use this cell to set up import statements for all of the packages that you}
         \PY{c+c1}{\PYZsh{}   plan to use.}
         
         \PY{c+c1}{\PYZsh{} Remember to include a \PYZsq{}magic word\PYZsq{} so that your visualizations are plotted}
         \PY{c+c1}{\PYZsh{}   inline with the notebook. See this page for more:}
         \PY{c+c1}{\PYZsh{}   http://ipython.readthedocs.io/en/stable/interactive/magics.html}
         \PY{k+kn}{import} \PY{n+nn}{pandas} \PY{k}{as} \PY{n+nn}{pd}
         \PY{k+kn}{import} \PY{n+nn}{numpy} \PY{k}{as} \PY{n+nn}{np}
         \PY{k+kn}{import} \PY{n+nn}{matplotlib}\PY{n+nn}{.}\PY{n+nn}{pyplot} \PY{k}{as} \PY{n+nn}{plt}
         \PY{o}{\PYZpc{}} \PY{n}{matplotlib} \PY{n}{inline}
         
         \PY{n}{df} \PY{o}{=} \PY{n}{pd}\PY{o}{.}\PY{n}{read\PYZus{}csv}\PY{p}{(}\PY{l+s+s1}{\PYZsq{}}\PY{l+s+s1}{movies.csv}\PY{l+s+s1}{\PYZsq{}}\PY{p}{,} \PY{n}{index\PYZus{}col}\PY{o}{=}\PY{l+s+s1}{\PYZsq{}}\PY{l+s+s1}{id}\PY{l+s+s1}{\PYZsq{}}\PY{p}{)}
\end{Verbatim}


     \#\# Data Wrangling

\begin{quote}
\textbf{Tip}: In this section of the report, you will load in the data,
check for cleanliness, and then trim and clean your dataset for
analysis. Make sure that you document your steps carefully and justify
your cleaning decisions.
\end{quote}

\subsubsection{General Properties}\label{general-properties}

    \begin{Verbatim}[commandchars=\\\{\}]
{\color{incolor}In [{\color{incolor}8}]:} \PY{c+c1}{\PYZsh{} Load your data and print out a few lines. Perform operations to inspect data}
        \PY{c+c1}{\PYZsh{}   types and look for instances of missing or possibly errant data.}
        \PY{n}{df}\PY{o}{.}\PY{n}{head}\PY{p}{(}\PY{l+m+mi}{2}\PY{p}{)}
\end{Verbatim}


\begin{Verbatim}[commandchars=\\\{\}]
{\color{outcolor}Out[{\color{outcolor}8}]:}           imdb\_id  popularity     budget     revenue      original\_title  \textbackslash{}
        id                                                                         
        135397  tt0369610   32.985763  150000000  1513528810      Jurassic World   
        76341   tt1392190   28.419936  150000000   378436354  Mad Max: Fury Road   
        
                                                             cast  \textbackslash{}
        id                                                          
        135397  Chris Pratt|Bryce Dallas Howard|Irrfan Khan|Vi{\ldots}   
        76341   Tom Hardy|Charlize Theron|Hugh Keays-Byrne|Nic{\ldots}   
        
                                     homepage         director             tagline  \textbackslash{}
        id                                                                           
        135397  http://www.jurassicworld.com/  Colin Trevorrow   The park is open.   
        76341     http://www.madmaxmovie.com/    George Miller  What a Lovely Day.   
        
                                                         keywords  \textbackslash{}
        id                                                          
        135397  monster|dna|tyrannosaurus rex|velociraptor|island   
        76341    future|chase|post-apocalyptic|dystopia|australia   
        
                                                         overview  runtime  \textbackslash{}
        id                                                                   
        135397  Twenty-two years after the events of Jurassic {\ldots}      124   
        76341   An apocalyptic story set in the furthest reach{\ldots}      120   
        
                                                   genres  \textbackslash{}
        id                                                  
        135397  Action|Adventure|Science Fiction|Thriller   
        76341   Action|Adventure|Science Fiction|Thriller   
        
                                             production\_companies release\_date  \textbackslash{}
        id                                                                       
        135397  Universal Studios|Amblin Entertainment|Legenda{\ldots}       6/9/15   
        76341   Village Roadshow Pictures|Kennedy Miller Produ{\ldots}      5/13/15   
        
                vote\_count  vote\_average  release\_year    budget\_adj   revenue\_adj  
        id                                                                          
        135397        5562           6.5          2015  1.379999e+08  1.392446e+09  
        76341         6185           7.1          2015  1.379999e+08  3.481613e+08  
\end{Verbatim}
            
    \begin{Verbatim}[commandchars=\\\{\}]
{\color{incolor}In [{\color{incolor}6}]:} \PY{c+c1}{\PYZsh{} Checking to see any columns missing data.}
        \PY{n}{df}\PY{o}{.}\PY{n}{info}\PY{p}{(}\PY{p}{)}
\end{Verbatim}


    \begin{Verbatim}[commandchars=\\\{\}]
<class 'pandas.core.frame.DataFrame'>
RangeIndex: 10866 entries, 0 to 10865
Data columns (total 21 columns):
id                      10866 non-null int64
imdb\_id                 10856 non-null object
popularity              10866 non-null float64
budget                  10866 non-null int64
revenue                 10866 non-null int64
original\_title          10866 non-null object
cast                    10790 non-null object
homepage                2936 non-null object
director                10822 non-null object
tagline                 8042 non-null object
keywords                9373 non-null object
overview                10862 non-null object
runtime                 10866 non-null int64
genres                  10843 non-null object
production\_companies    9836 non-null object
release\_date            10866 non-null object
vote\_count              10866 non-null int64
vote\_average            10866 non-null float64
release\_year            10866 non-null int64
budget\_adj              10866 non-null float64
revenue\_adj             10866 non-null float64
dtypes: float64(4), int64(6), object(11)
memory usage: 1.7+ MB

    \end{Verbatim}

    \begin{quote}
\textbf{Tip}: You should \emph{not} perform too many operations in each
cell. Create cells freely to explore your data. One option that you can
take with this project is to do a lot of explorations in an initial
notebook. These don't have to be organized, but make sure you use enough
comments to understand the purpose of each code cell. Then, after you're
done with your analysis, create a duplicate notebook where you will trim
the excess and organize your steps so that you have a flowing, cohesive
report.
\end{quote}

\begin{quote}
\textbf{Tip}: Make sure that you keep your reader informed on the steps
that you are taking in your investigation. Follow every code cell, or
every set of related code cells, with a markdown cell to describe to the
reader what was found in the preceding cell(s). Try to make it so that
the reader can then understand what they will be seeing in the following
cell(s).
\end{quote}

\subsubsection{Data Cleaning (I will delete unnecessary columns and
check data
type)}\label{data-cleaning-i-will-delete-unnecessary-columns-and-check-data-type}

    \begin{Verbatim}[commandchars=\\\{\}]
{\color{incolor}In [{\color{incolor}15}]:} \PY{c+c1}{\PYZsh{} After discussing the structure of the data and any problems that need to be}
         \PY{c+c1}{\PYZsh{}   cleaned, perform those cleaning steps in the second part of this section.}
         
         \PY{c+c1}{\PYZsh{}After checking the data, I see I dont need columns : imdb\PYZus{}id, cast, homepage, director, tagline, keywords, overview, production\PYZus{}companies, budget,revenue,genres}
         \PY{n}{df}\PY{o}{.}\PY{n}{drop}\PY{p}{(}\PY{p}{[}\PY{l+s+s1}{\PYZsq{}}\PY{l+s+s1}{imdb\PYZus{}id}\PY{l+s+s1}{\PYZsq{}}\PY{p}{,} \PY{l+s+s1}{\PYZsq{}}\PY{l+s+s1}{cast}\PY{l+s+s1}{\PYZsq{}}\PY{p}{,} \PY{l+s+s1}{\PYZsq{}}\PY{l+s+s1}{homepage}\PY{l+s+s1}{\PYZsq{}}\PY{p}{,} \PY{l+s+s1}{\PYZsq{}}\PY{l+s+s1}{director}\PY{l+s+s1}{\PYZsq{}}\PY{p}{,} \PY{l+s+s1}{\PYZsq{}}\PY{l+s+s1}{tagline}\PY{l+s+s1}{\PYZsq{}}\PY{p}{,} \PY{l+s+s1}{\PYZsq{}}\PY{l+s+s1}{keywords}\PY{l+s+s1}{\PYZsq{}}\PY{p}{,} \PY{l+s+s1}{\PYZsq{}}\PY{l+s+s1}{overview}\PY{l+s+s1}{\PYZsq{}}\PY{p}{,} \PY{l+s+s1}{\PYZsq{}}\PY{l+s+s1}{production\PYZus{}companies}\PY{l+s+s1}{\PYZsq{}}\PY{p}{,}\PY{l+s+s1}{\PYZsq{}}\PY{l+s+s1}{budget}\PY{l+s+s1}{\PYZsq{}}\PY{p}{,}\PY{l+s+s1}{\PYZsq{}}\PY{l+s+s1}{revenue}\PY{l+s+s1}{\PYZsq{}}\PY{p}{,}\PY{l+s+s1}{\PYZsq{}}\PY{l+s+s1}{genres}\PY{l+s+s1}{\PYZsq{}}\PY{p}{]}\PY{p}{,} \PY{n}{axis}\PY{o}{=}\PY{l+m+mi}{1}\PY{p}{,} \PY{n}{inplace}\PY{o}{=}\PY{k+kc}{True}\PY{p}{)}
         \PY{n}{df}\PY{o}{.}\PY{n}{head}\PY{p}{(}\PY{l+m+mi}{1}\PY{p}{)}
\end{Verbatim}


\begin{Verbatim}[commandchars=\\\{\}]
{\color{outcolor}Out[{\color{outcolor}15}]:}         popularity  original\_title  runtime release\_date  vote\_count  \textbackslash{}
         id                                                                     
         135397   32.985763  Jurassic World      124       6/9/15        5562   
         
                 vote\_average  release\_year    budget\_adj   revenue\_adj  
         id                                                              
         135397           6.5          2015  1.379999e+08  1.392446e+09  
\end{Verbatim}
            
    \begin{Verbatim}[commandchars=\\\{\}]
{\color{incolor}In [{\color{incolor}17}]:} \PY{n}{df}\PY{o}{.}\PY{n}{info}\PY{p}{(}\PY{p}{)}
\end{Verbatim}


    \begin{Verbatim}[commandchars=\\\{\}]
<class 'pandas.core.frame.DataFrame'>
Int64Index: 10866 entries, 135397 to 22293
Data columns (total 9 columns):
popularity        10866 non-null float64
original\_title    10866 non-null object
runtime           10866 non-null int64
release\_date      10866 non-null object
vote\_count        10866 non-null int64
vote\_average      10866 non-null float64
release\_year      10866 non-null int64
budget\_adj        10866 non-null float64
revenue\_adj       10866 non-null float64
dtypes: float64(4), int64(3), object(2)
memory usage: 848.9+ KB

    \end{Verbatim}

    \begin{Verbatim}[commandchars=\\\{\}]
{\color{incolor}In [{\color{incolor}18}]:} \PY{n}{df}\PY{p}{[}\PY{l+s+s1}{\PYZsq{}}\PY{l+s+s1}{release\PYZus{}date}\PY{l+s+s1}{\PYZsq{}}\PY{p}{]}\PY{o}{=}\PY{n}{pd}\PY{o}{.}\PY{n}{to\PYZus{}datetime}\PY{p}{(}\PY{n}{df}\PY{p}{[}\PY{l+s+s1}{\PYZsq{}}\PY{l+s+s1}{release\PYZus{}date}\PY{l+s+s1}{\PYZsq{}}\PY{p}{]}\PY{p}{)}
\end{Verbatim}


    \begin{Verbatim}[commandchars=\\\{\}]
{\color{incolor}In [{\color{incolor}27}]:} \PY{n}{df}\PY{o}{.}\PY{n}{head}\PY{p}{(}\PY{p}{)}
\end{Verbatim}


\begin{Verbatim}[commandchars=\\\{\}]
{\color{outcolor}Out[{\color{outcolor}27}]:}         popularity                original\_title  runtime release\_date  \textbackslash{}
         id                                                                       
         135397   32.985763                Jurassic World      124   2015-06-09   
         76341    28.419936            Mad Max: Fury Road      120   2015-05-13   
         262500   13.112507                     Insurgent      119   2015-03-18   
         140607   11.173104  Star Wars: The Force Awakens      136   2015-12-15   
         168259    9.335014                     Furious 7      137   2015-04-01   
         
                 vote\_count  vote\_average  release\_year    budget\_adj   revenue\_adj  
         id                                                                          
         135397        5562           6.5          2015  1.379999e+08  1.392446e+09  
         76341         6185           7.1          2015  1.379999e+08  3.481613e+08  
         262500        2480           6.3          2015  1.012000e+08  2.716190e+08  
         140607        5292           7.5          2015  1.839999e+08  1.902723e+09  
         168259        2947           7.3          2015  1.747999e+08  1.385749e+09  
\end{Verbatim}
            
     \#\# Exploratory Data Analysis

\begin{quote}
\textbf{Tip}: Now that you've trimmed and cleaned your data, you're
ready to move on to exploration. Compute statistics and create
visualizations with the goal of addressing the research questions that
you posed in the Introduction section. It is recommended that you be
systematic with your approach. Look at one variable at a time, and then
follow it up by looking at relationships between variables.
\end{quote}

\subsubsection{Research Question 1 (How much money do producers spend on
average by
years?)}\label{research-question-1-how-much-money-do-producers-spend-on-average-by-years}

    \begin{Verbatim}[commandchars=\\\{\}]
{\color{incolor}In [{\color{incolor}62}]:} \PY{n}{pd}\PY{o}{.}\PY{n}{plotting}\PY{o}{.}\PY{n}{scatter\PYZus{}matrix}\PY{p}{(}\PY{n}{df}\PY{p}{,}\PY{n}{figsize}\PY{o}{=}\PY{p}{(}\PY{l+m+mi}{15}\PY{p}{,}\PY{l+m+mi}{15}\PY{p}{)}\PY{p}{)}
\end{Verbatim}


\begin{Verbatim}[commandchars=\\\{\}]
{\color{outcolor}Out[{\color{outcolor}62}]:} array([[<matplotlib.axes.\_subplots.AxesSubplot object at 0x11662fd68>,
                 <matplotlib.axes.\_subplots.AxesSubplot object at 0x11667f278>,
                 <matplotlib.axes.\_subplots.AxesSubplot object at 0x116658be0>,
                 <matplotlib.axes.\_subplots.AxesSubplot object at 0x116720278>,
                 <matplotlib.axes.\_subplots.AxesSubplot object at 0x11678f470>,
                 <matplotlib.axes.\_subplots.AxesSubplot object at 0x11678f048>,
                 <matplotlib.axes.\_subplots.AxesSubplot object at 0x11684ea58>],
                [<matplotlib.axes.\_subplots.AxesSubplot object at 0x116896160>,
                 <matplotlib.axes.\_subplots.AxesSubplot object at 0x1168c85c0>,
                 <matplotlib.axes.\_subplots.AxesSubplot object at 0x1168582b0>,
                 <matplotlib.axes.\_subplots.AxesSubplot object at 0x1169438d0>,
                 <matplotlib.axes.\_subplots.AxesSubplot object at 0x1169874e0>,
                 <matplotlib.axes.\_subplots.AxesSubplot object at 0x1169a6a90>,
                 <matplotlib.axes.\_subplots.AxesSubplot object at 0x116db0be0>],
                [<matplotlib.axes.\_subplots.AxesSubplot object at 0x124c13f28>,
                 <matplotlib.axes.\_subplots.AxesSubplot object at 0x112ce3550>,
                 <matplotlib.axes.\_subplots.AxesSubplot object at 0x116b30278>,
                 <matplotlib.axes.\_subplots.AxesSubplot object at 0x1169e1c50>,
                 <matplotlib.axes.\_subplots.AxesSubplot object at 0x116a1ccc0>,
                 <matplotlib.axes.\_subplots.AxesSubplot object at 0x123c0b4e0>,
                 <matplotlib.axes.\_subplots.AxesSubplot object at 0x116b0e240>],
                [<matplotlib.axes.\_subplots.AxesSubplot object at 0x116bd50f0>,
                 <matplotlib.axes.\_subplots.AxesSubplot object at 0x123c15518>,
                 <matplotlib.axes.\_subplots.AxesSubplot object at 0x118f1b278>,
                 <matplotlib.axes.\_subplots.AxesSubplot object at 0x124c46860>,
                 <matplotlib.axes.\_subplots.AxesSubplot object at 0x118f872b0>,
                 <matplotlib.axes.\_subplots.AxesSubplot object at 0x118fd6ba8>,
                 <matplotlib.axes.\_subplots.AxesSubplot object at 0x118ff5be0>],
                [<matplotlib.axes.\_subplots.AxesSubplot object at 0x11903a588>,
                 <matplotlib.axes.\_subplots.AxesSubplot object at 0x121a56978>,
                 <matplotlib.axes.\_subplots.AxesSubplot object at 0x1190bca20>,
                 <matplotlib.axes.\_subplots.AxesSubplot object at 0x118ffee48>,
                 <matplotlib.axes.\_subplots.AxesSubplot object at 0x1247d1b00>,
                 <matplotlib.axes.\_subplots.AxesSubplot object at 0x119159cf8>,
                 <matplotlib.axes.\_subplots.AxesSubplot object at 0x121a6e588>],
                [<matplotlib.axes.\_subplots.AxesSubplot object at 0x1191dc588>,
                 <matplotlib.axes.\_subplots.AxesSubplot object at 0x1192179b0>,
                 <matplotlib.axes.\_subplots.AxesSubplot object at 0x119254828>,
                 <matplotlib.axes.\_subplots.AxesSubplot object at 0x119291390>,
                 <matplotlib.axes.\_subplots.AxesSubplot object at 0x121a87390>,
                 <matplotlib.axes.\_subplots.AxesSubplot object at 0x1190d99b0>,
                 <matplotlib.axes.\_subplots.AxesSubplot object at 0x119334128>],
                [<matplotlib.axes.\_subplots.AxesSubplot object at 0x121a94e48>,
                 <matplotlib.axes.\_subplots.AxesSubplot object at 0x1193ac080>,
                 <matplotlib.axes.\_subplots.AxesSubplot object at 0x1193f5eb8>,
                 <matplotlib.axes.\_subplots.AxesSubplot object at 0x124db1828>,
                 <matplotlib.axes.\_subplots.AxesSubplot object at 0x119464198>,
                 <matplotlib.axes.\_subplots.AxesSubplot object at 0x121aac898>,
                 <matplotlib.axes.\_subplots.AxesSubplot object at 0x1194dd9e8>]], dtype=object)
\end{Verbatim}
            
    \begin{center}
    \adjustimage{max size={0.9\linewidth}{0.9\paperheight}}{output_13_1.png}
    \end{center}
    { \hspace*{\fill} \\}
    
    \begin{Verbatim}[commandchars=\\\{\}]
{\color{incolor}In [{\color{incolor}45}]:} \PY{n}{df\PYZus{}year} \PY{o}{=} \PY{n}{df}\PY{o}{.}\PY{n}{loc}\PY{p}{[}\PY{p}{:}\PY{p}{,}\PY{l+s+s1}{\PYZsq{}}\PY{l+s+s1}{release\PYZus{}year}\PY{l+s+s1}{\PYZsq{}}\PY{p}{:}\PY{l+s+s1}{\PYZsq{}}\PY{l+s+s1}{revenue\PYZus{}adj}\PY{l+s+s1}{\PYZsq{}}\PY{p}{]}
         \PY{n}{df\PYZus{}year}\PY{o}{.}\PY{n}{head}\PY{p}{(}\PY{p}{)}
\end{Verbatim}


\begin{Verbatim}[commandchars=\\\{\}]
{\color{outcolor}Out[{\color{outcolor}45}]:}         release\_year    budget\_adj   revenue\_adj
         id                                              
         135397          2015  1.379999e+08  1.392446e+09
         76341           2015  1.379999e+08  3.481613e+08
         262500          2015  1.012000e+08  2.716190e+08
         140607          2015  1.839999e+08  1.902723e+09
         168259          2015  1.747999e+08  1.385749e+09
\end{Verbatim}
            
    \begin{Verbatim}[commandchars=\\\{\}]
{\color{incolor}In [{\color{incolor}61}]:} \PY{n}{df\PYZus{}year\PYZus{}mean} \PY{o}{=} \PY{n}{df}\PY{o}{.}\PY{n}{groupby}\PY{p}{(}\PY{p}{[}\PY{l+s+s1}{\PYZsq{}}\PY{l+s+s1}{release\PYZus{}year}\PY{l+s+s1}{\PYZsq{}}\PY{p}{]}\PY{p}{)}\PY{o}{.}\PY{n}{mean}\PY{p}{(}\PY{p}{)}
         \PY{n}{df\PYZus{}year\PYZus{}mean}\PY{o}{.}\PY{n}{head}\PY{p}{(}\PY{p}{)}
\end{Verbatim}


\begin{Verbatim}[commandchars=\\\{\}]
{\color{outcolor}Out[{\color{outcolor}61}]:}               popularity     runtime  vote\_count  vote\_average    budget\_adj  \textbackslash{}
         release\_year                                                                   
         1960            0.458932  110.656250   77.531250      6.325000  5.082036e+06   
         1961            0.422827  119.419355   77.580645      6.374194  1.085687e+07   
         1962            0.454783  124.343750   74.750000      6.343750  1.232693e+07   
         1963            0.502706  111.323529   82.823529      6.329412  1.535687e+07   
         1964            0.412428  109.214286   74.690476      6.211905  6.608980e+06   
         
                        revenue\_adj  
         release\_year                
         1960          3.340991e+07  
         1961          7.947167e+07  
         1962          4.856238e+07  
         1963          3.924580e+07  
         1964          5.707603e+07  
\end{Verbatim}
            
    \begin{Verbatim}[commandchars=\\\{\}]
{\color{incolor}In [{\color{incolor}69}]:} \PY{n}{plt}\PY{o}{.}\PY{n}{subplots}\PY{p}{(}\PY{n}{figsize}\PY{o}{=}\PY{p}{(}\PY{l+m+mi}{8}\PY{p}{,}\PY{l+m+mi}{5}\PY{p}{)}\PY{p}{)}
         \PY{n}{plt}\PY{o}{.}\PY{n}{bar}\PY{p}{(}\PY{n}{df\PYZus{}year}\PY{p}{[}\PY{l+s+s1}{\PYZsq{}}\PY{l+s+s1}{release\PYZus{}year}\PY{l+s+s1}{\PYZsq{}}\PY{p}{]}\PY{p}{,}\PY{n}{df\PYZus{}year}\PY{p}{[}\PY{l+s+s1}{\PYZsq{}}\PY{l+s+s1}{budget\PYZus{}adj}\PY{l+s+s1}{\PYZsq{}}\PY{p}{]}\PY{p}{)}
         \PY{n}{plt}\PY{o}{.}\PY{n}{title}\PY{p}{(}\PY{l+s+s1}{\PYZsq{}}\PY{l+s+s1}{Spending by years}\PY{l+s+s1}{\PYZsq{}}\PY{p}{)}
         \PY{n}{plt}\PY{o}{.}\PY{n}{xlabel}\PY{p}{(}\PY{l+s+s1}{\PYZsq{}}\PY{l+s+s1}{Year}\PY{l+s+s1}{\PYZsq{}}\PY{p}{)}
         \PY{n}{plt}\PY{o}{.}\PY{n}{ylabel}\PY{p}{(}\PY{l+s+s1}{\PYZsq{}}\PY{l+s+s1}{Dollars}\PY{l+s+s1}{\PYZsq{}}\PY{p}{)}\PY{p}{;}
\end{Verbatim}


    \begin{center}
    \adjustimage{max size={0.9\linewidth}{0.9\paperheight}}{output_16_0.png}
    \end{center}
    { \hspace*{\fill} \\}
    
    \begin{Verbatim}[commandchars=\\\{\}]
{\color{incolor}In [{\color{incolor}47}]:} \PY{n}{plt}\PY{o}{.}\PY{n}{subplots}\PY{p}{(}\PY{n}{figsize}\PY{o}{=}\PY{p}{(}\PY{l+m+mi}{8}\PY{p}{,}\PY{l+m+mi}{5}\PY{p}{)}\PY{p}{)}
         \PY{n}{plt}\PY{o}{.}\PY{n}{bar}\PY{p}{(}\PY{n}{df\PYZus{}year}\PY{p}{[}\PY{l+s+s1}{\PYZsq{}}\PY{l+s+s1}{release\PYZus{}year}\PY{l+s+s1}{\PYZsq{}}\PY{p}{]}\PY{p}{,}\PY{n}{df\PYZus{}year}\PY{p}{[}\PY{l+s+s1}{\PYZsq{}}\PY{l+s+s1}{revenue\PYZus{}adj}\PY{l+s+s1}{\PYZsq{}}\PY{p}{]}\PY{p}{)}
         \PY{n}{plt}\PY{o}{.}\PY{n}{title}\PY{p}{(}\PY{l+s+s1}{\PYZsq{}}\PY{l+s+s1}{Revenue by years}\PY{l+s+s1}{\PYZsq{}}\PY{p}{)}
         \PY{n}{plt}\PY{o}{.}\PY{n}{xlabel}\PY{p}{(}\PY{l+s+s1}{\PYZsq{}}\PY{l+s+s1}{Year}\PY{l+s+s1}{\PYZsq{}}\PY{p}{)}
         \PY{n}{plt}\PY{o}{.}\PY{n}{ylabel}\PY{p}{(}\PY{l+s+s1}{\PYZsq{}}\PY{l+s+s1}{Dollars}\PY{l+s+s1}{\PYZsq{}}\PY{p}{)}\PY{p}{;}
\end{Verbatim}


    \begin{center}
    \adjustimage{max size={0.9\linewidth}{0.9\paperheight}}{output_17_0.png}
    \end{center}
    { \hspace*{\fill} \\}
    
    \subsubsection{Research Question 2 (What month do movies release the
most?)}\label{research-question-2-what-month-do-movies-release-the-most}

    \begin{Verbatim}[commandchars=\\\{\}]
{\color{incolor}In [{\color{incolor}65}]:} \PY{c+c1}{\PYZsh{} Continue to explore the data to address your additional research}
         \PY{c+c1}{\PYZsh{}   questions. Add more headers as needed if you have more questions to}
         \PY{c+c1}{\PYZsh{}   investigate.}
         
         \PY{c+c1}{\PYZsh{} I get the month of the release day, then plot them to see in what month we have }
         \PY{n}{df\PYZus{}month} \PY{o}{=} \PY{n}{df}\PY{p}{[}\PY{l+s+s1}{\PYZsq{}}\PY{l+s+s1}{release\PYZus{}date}\PY{l+s+s1}{\PYZsq{}}\PY{p}{]}\PY{o}{.}\PY{n}{dt}\PY{o}{.}\PY{n}{month}
         \PY{n}{df\PYZus{}month}\PY{o}{.}\PY{n}{head}\PY{p}{(}\PY{p}{)}
\end{Verbatim}


\begin{Verbatim}[commandchars=\\\{\}]
{\color{outcolor}Out[{\color{outcolor}65}]:} id
         135397     6
         76341      5
         262500     3
         140607    12
         168259     4
         Name: release\_date, dtype: int64
\end{Verbatim}
            
    \begin{Verbatim}[commandchars=\\\{\}]
{\color{incolor}In [{\color{incolor}68}]:} \PY{c+c1}{\PYZsh{} plot data   .... I dont know how to make them in order.  I tried to use numpy but it\PYZsq{}s not work for me}
         \PY{n}{df\PYZus{}month}\PY{o}{.}\PY{n}{value\PYZus{}counts}\PY{p}{(}\PY{p}{)}\PY{o}{.}\PY{n}{plot}\PY{p}{(}\PY{n}{kind}\PY{o}{=}\PY{l+s+s1}{\PYZsq{}}\PY{l+s+s1}{bar}\PY{l+s+s1}{\PYZsq{}}\PY{p}{)}
\end{Verbatim}


\begin{Verbatim}[commandchars=\\\{\}]
{\color{outcolor}Out[{\color{outcolor}68}]:} <matplotlib.axes.\_subplots.AxesSubplot at 0x1199a29b0>
\end{Verbatim}
            
    \begin{center}
    \adjustimage{max size={0.9\linewidth}{0.9\paperheight}}{output_20_1.png}
    \end{center}
    { \hspace*{\fill} \\}
    
     \#\# Conclusions

\begin{quote}
\textbf{Question 1: How much money do producers spend on average by
years?}: Looking the plots, I see that producers spend more money to
make a new movie. However, I dont see the revenue increases much along
with the money they spend.
\end{quote}

\begin{quote}
\textbf{Question 2: What month do movies release the most?}: As we can
see, September is the month that new movies come out the most
\end{quote}


    % Add a bibliography block to the postdoc
    
    
    
    \end{document}
